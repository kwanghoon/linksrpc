\documentclass[a4paper]{article}

\RequirePackage[undo-recent-deprecations]{expl3} %% Fix: latex3 problem with ebproof

\usepackage[pages=all, color=black, position={current page.south}, placement=bottom, scale=1, opacity=1, vshift=5mm]{background}
%% \SetBgContents{
%% 	\tt This work is shared under a \href{https://creativecommons.org/licenses/by-sa/4.0/}{CC BY-SA 4.0 license} unless otherwise noted
%% }      % copyright

\usepackage[margin=1in]{geometry} % full-width

% AMS Packages
\usepackage{amsmath}
\usepackage{amsthm}
\usepackage{amssymb}

% Unicode
\usepackage[utf8]{inputenc}
\usepackage{hyperref}
\hypersetup{
	unicode,
%	colorlinks,
%	breaklinks,
%	urlcolor=cyan, 
%	linkcolor=blue, 
	pdfauthor={Author One, Author Two, Author Three},
	pdftitle={A simple article template},
	pdfsubject={A simple article template},
	pdfkeywords={article, template, simple},
	pdfproducer={LaTeX},
	pdfcreator={pdflatex}
}

% Vietnamese
%\usepackage{vntex}

% Natbib
\usepackage[sort&compress,numbers,square]{natbib}
%\bibliographystyle{mplainnat}
\bibliographystyle{plain}
  
% Theorem, Lemma, etc
\theoremstyle{plain}
\newtheorem{theorem}{Theorem}
\newtheorem{corollary}[theorem]{Corollary}
\newtheorem{lemma}[theorem]{Lemma}
\newtheorem{claim}{Claim}[theorem]
\newtheorem{axiom}[theorem]{Axiom}
\newtheorem{conjecture}[theorem]{Conjecture}
\newtheorem{fact}[theorem]{Fact}
\newtheorem{hypothesis}[theorem]{Hypothesis}
\newtheorem{assumption}[theorem]{Assumption}
\newtheorem{proposition}[theorem]{Proposition}
\newtheorem{criterion}[theorem]{Criterion}
\theoremstyle{definition}
\newtheorem{definition}[theorem]{Definition}
\newtheorem{example}[theorem]{Example}
\newtheorem{remark}[theorem]{Remark}
\newtheorem{problem}[theorem]{Problem}
\newtheorem{principle}[theorem]{Principle}

%%% ebproof column width
\newcommand{\rulewidth}{.8\linewidth}
\newcommand{\ruleverticalsep}{0.5cm}

\usepackage{graphicx, color}
\graphicspath{{fig/}}

%\usepackage[linesnumbered,ruled,vlined,commentsnumbered]{algorithm2e} % use algorithm2e for typesetting algorithms
\usepackage{algorithm, algpseudocode} % use algorithm and algorithmicx for typesetting algorithms
\usepackage{mathrsfs} % for \mathscr command

\usepackage{lipsum}

%%% RPC Calculi related commands %%%

%%
\usepackage{listings}
\lstset
{
    basicstyle=\footnotesize\ttfamily,
    numbers=left,
    stepnumber=1,
    frame=single,
}
\usepackage{ebproof}
\usepackage{multicol}
\usepackage{xspace}


%% Macros for General Inference Rules
\newcommand{\rpc}{$\lambda_{rpc}$\xspace}
\newcommand{\typedrpc}{$\lambda_{rpc}^{typed}$\xspace}
\newcommand{\polyrpc}{$\lambda_{rpc}^{\forall}$\xspace}
\newcommand{\linksrpc}{$\lambda_{rpc}^{Links}$\xspace}
\newcommand{\kindedpolyrpc}{$\lambda_{rpc}^{\forall,k}$\xspace}

\newcommand{\stateencrpc}{$\lambda_{rpc}^{enc}$\xspace}
\newcommand{\statefulrpc}{$\lambda_{rpc}^{state}$\xspace}

\newcommand{\cs}{$\lambda_{cs}$\xspace}
\newcommand{\polycs}{$\lambda_{cs}^{\forall}$\xspace}
\newcommand{\stateenccs}{$\lambda_{cs}^{enc}$\xspace}
\newcommand{\statefulcs}{$\lambda_{cs}^{state}$\xpace}

\newcommand{\client}{\textbf{c}}
\newcommand{\server}{\textbf{s}}
\newcommand{\clientserver}{\textbf{cs}}

\newcommand{\statickind}{sta}
\newcommand{\dynamickind}{dyn}

\newcommand{\evalRPC}[3]{#1\Downarrow_{#2}#3}
\newcommand{\evalRPCC}[2]{#1\Downarrow_{\client}#2}
\newcommand{\evalRPCS}[2]{#1\Downarrow_{\server}#2}
\newcommand{\lamL}[3]{\lambda^{#1}#2.#3}
\newcommand{\appL}[3]{#1{\ }^{#2}#3}
\newcommand{\subst}[2]{\{#1/#2\}}
\newcommand{\llet}[3]{\textsf{let} \ #1 = #2 \ \textsf{in} \ #3}
\newcommand{\ldokeyword}{\textsf{do}}
\newcommand{\ldo}[3]{\textsf{do} \ #1 \leftarrow #2 \ \textsf{in} \ #3}
\newcommand{\ldovoid}[2]{\textsf{do} \ #1 \ \textsf{in} \ #2}
\newcommand{\lunit}[1]{\textsf{unit} \ #1}

\newcommand{\textsfGen}{\textsf{gen}}
\newcommand{\gen}[3]{\textsfGen(#1,#2,#3)}

\newcommand{\textsfReq}{\textsf{req}}
\newcommand{\req}[2]{\textsfReq(#1,#2)}

\newcommand{\textsfCall}{\textsf{call}}
\newcommand{\call}[2]{\textsfCall(#1,#2)}

\newcommand{\textsfRet}{\textsf{ret}}
\newcommand{\ret}[1]{\textsfRet(#1)}

\newcommand{\textsfCase}{\textsf{case}}
\newcommand{\textsfOf}{\textsf{of}}
\newcommand{\case}[2]{\textsfCase ~ #1 ~\textsfOf ~ #2}

\newcommand{\fun}{\rightarrow}
\newcommand{\funL}[1]{\xrightarrow{#1}}
\newcommand{\funLC}[2]{\xrightarrow[#2]{#1}}
\newcommand{\tyenv}{\Gamma}
\newcommand{\tyenvExt}[2]{\Gamma,#1:#2}
\newcommand{\tyenvExtWith}[1]{\Gamma,#1}
\newcommand{\varenv}{\Delta}
\newcommand{\varenvExt}[1]{\Delta,#1}
%\newcommand{\typing}[4]{#1\rhd_{#2} #3 : #4}
\newcommand{\typing}[4]{#1\vdash_{#2} #3 : #4}
\newcommand{\kinding}[3]{#1\vdash #2 : #3}
\newcommand{\funtyping}[5]{#1\vdash #4 : #5}     % ;#2, #3
\newcommand{\codetyping}[4]{#1\vdash_{code} #3 : #4}  % _{#2}
\newcommand{\polytyping}[5]{#1;#2\vdash_{#3} #4 : #5}
\newcommand{\conftyping}[2]{\vdash #1 : #2}
\newcommand{\stacktyping}[3]{\vdash_{#1} #2 : #3}
\newcommand{\fvtyping}[2]{#1\vdash #2}

\newcommand{\typingBlack}[4]{#1\blacktriangleright_{#2} #3 : #4}

\newcommand{\loceta}[2]{{#1}\rightsquigarrow{#2}}

\newcommand{\enc}{\textsf{enc}}
\newcommand{\evalStateEncRPCC}[2]{$#1\Downarrow_{\client}^{\enc}#2$}
\newcommand{\evalStateEncRPCS}[3]{$#1;#2\Downarrow_{\server}^{\enc}#3$}

\newcommand{\sta}{\textsf{state}}
\newcommand{\evalStatefulRPCC}[3]{\evalStatefulRPC{#1}{#2}{\client}{#3}}
\newcommand{\evalStatefulRPCS}[3]{\evalStatefulRPC{#1}{#2}{\server}{#3}}
\newcommand{\evalStatefulRPC}[4]{${#1};{#2}\Downarrow_{#3}^{\sta}{#4}$}

\newcommand{\deep}{\textsf{deep}}
\newcommand{\evalDeeplyStatefulRPCC}[3]{\evalDeeplyStatefulRPC{#1}{#2}{\client}{#3}}
\newcommand{\evalDeeplyStatefulRPCS}[3]{\evalDeeplyStatefulRPC{#1}{#2}{\server}{#3}}
\newcommand{\evalDeeplyStatefulRPC}[4]{${#1};{#2}\Downarrow_{#3}^{\deep}{#4}$}

\newcommand{\IdK}{\textsf{Id}}
\newcommand{\FunK}[3]{\textsf{Fun} \ #2 \ #3}
\newcommand{\AppK}[3]{\textsf{App} \ #2 \ #3}
%\newcommand{\FunK}[3]{\textsf{Fun}^{#1} \ #2 \ #3}
%\newcommand{\AppK}[3]{\textsf{App}^{#1} \ #2 \ #3}

\newcommand{\reify}[1]{\ulcorner #1 \urcorner}

\newcommand{\RightarrowEnc}{\Rightarrow^{enc}}
\newcommand{\RightarrowEncStar}{\Rightarrow^{enc*}}
\newcommand{\RightarrowEncPlus}{\Rightarrow^{enc+}}

\newcommand{\runStateEncRPC}[2]{$#1 \RightarrowEnc #2$}
\newcommand{\runStateEncRPCStar}[2]{$#1 \RightarrowEncStar #2$}
\newcommand{\runStateEncRPCPlus}[2]{$#1 \RightarrowEncPlus #2$}

\newcommand{\runStateEncCS}[2]{$#1 \Rightarrow^{enc} #2$}
\newcommand{\runStateEncCSStar}[2]{$#1 \Rightarrow^{enc*} #2$}

\newcommand{\runStatefulRPC}[2]{$#1 \Rightarrow^{state} #2$}
\newcommand{\runStatefulRPCStar}[2]{$#1 \Rightarrow^{state*} #2$}

\newcommand{\runStatefulCS}[2]{$#1 \Rightarrow^{state} #2$}
\newcommand{\runStatefulCSStar}[2]{$#1 \Rightarrow^{state*} #2$}

\newcommand{\emp}{\epsilon}
\newcommand{\substzsxs}{\{\bar{v}/\bar{z},\overline{w}/\bar{x} \}}
\newcommand{\substxs}{\{\overline{w}/\bar{x} \}}

\newcommand{\substZsXs}{\{\overline{V}/\bar{z},\overline{W}/\bar{x} \}}
\newcommand{\substXs}{\{\overline{W}/\bar{x} \}}

\newcommand{\LetK}[2]{\textsf{ctx}\ #1 \ #2}
%\newcommand{\LetK}[2]{(#1,#2)}
\newcommand{\opt}[1]{#1_{opt}}

\newcommand{\overlineK}{\overline{K}}
\newcommand{\overlinePi}{\overline{\Pi}}
\newcommand{\overlineDelta}{\overline{\Delta}}

%\newcommand{\comp}[1]{\rightsquigarrow_{#1}}
%\newcommand{\comps}{\rightsquigarrow_{\server}}
%\newcommand{\compc}{\rightsquigarrow_{\client}}
% \newcommand{\ccomp}[1]{\mathcal{C}[\![#1]\!]}
\newcommand{\ccomp}[1]{\mathcal{C}\llbracket#1\rrbracket}
%\newcommand{\scomp}[1]{S[\![#1]\!]}
% \newcommand{\vcomp}[1]{\mathcal{V}[\![#1]\!]}
\newcommand{\vcomp}[1]{\mathcal{V}\llbracket#1\rrbracket}
%\newcommand{\cconv}[1]{CC[\![#1]\!]}
%\newcommand{\cconvprg}[1]{CC_{prg}[\![#1]\!]}
\newcommand{\typeinf}[1]{G[\![#1]\!]}

% \newcommand{\ecomp}[1]{[\![#1]\!]}
\newcommand{\ecomp}[1]{\llbracket#1\rrbracket}

\newcommand{\FUNS}{\Phi}
\newcommand{\funstore}{\FUNS}
\newcommand{\funtype}{\funstore_{type}}
\newcommand{\funcode}{\funstore}
\newcommand{\fv}[1]{\textsf{fv}(#1)}
\newcommand{\dom}[1]{\textsf{dom}(#1)}
\newcommand{\clo}[2]{clo({#1},{#2})}
\newcommand{\cloty}[1]{Clo(#1)}
\newcommand{\valty}[1]{relocatable(#1)}

\newcommand{\Loc}{Loc}

\newcommand{\sessionNothing}{\makebox[0.3cm][c]{\scriptsize $nothing$}}
\newcommand{\sessionSomething}{\makebox[0.3cm][c]{\scriptsize $session$}}
\newcommand{\sessionOption}{\makebox[0.3cm][c]{\scriptsize $optSession$}}

\newcommand{\mono}[1]{[\![#1]\!]}

\newcommand{\optrpc}[1]{\mathcal{O}[\![#1]\!]}

\newcommand{\resolve}[2]{\{\!\{#1\}\!\}^{#2}}

\newcommand{\stack}{\Delta}
\newcommand{\emptystack}{\epsilon}

\newcommand{\conf}{\Sigma}
\newcommand{\confcs}[2]{\langle #1 | #2 \rangle}
\newcommand{\stackcsWith}[2]{ #1  |  #2 }
\newcommand{\stackcs}{\stackcsWith{ \stack_\client }{ \stack_\server }}
\newcommand{\confuntyped}{\sigma}

\newcommand{\run}{\longrightarrow}
\newcommand{\runequiv}{\Longrightarrow}
\newcommand{\structeqv}{\equiv}

\newcommand{\loctyvars}{\overline{l} \ \overline{\alpha}}
\newcommand{\loctys}{\overline{\Loc}\ \overline{A}}

\newcommand{\CloCodeType}{Ty}
\newcommand{\CloCode}{Code}
\newcommand{\OpenCode}{OpenCode}

%%%%%%%%%%%%%%%%%%%%%%%%%%%%%%%%%%%%



% Author info
\title{What is a title?}
\author{Kwanghoon Choi}
%\author{Author One$^1$\thanks{Author One was partially supported by Grant XXX} \and Author Two$^2$ \and Author Three$^1$}

\date{
  Chonnam National University, Gwangju, Republic of Korea
  \\ \texttt{kwanghoon.choi@jnu.ac.kr}\\[2ex]%
	%% $^1$Organization 1 \\ \texttt{\{auth1, auth3\}@org1.edu}\\%
	%% $^2$Organization 2 \\ \texttt{auth3@inst2.edu}\\[2ex]%
%	\today
}

\begin{document}

\maketitle
	
\begin{abstract}
This summarizes this paper...
		
\noindent\textbf{Keywords:} Location inference, Links, RPC calculus
\end{abstract}

%\tableofcontents
	
\section{Introduction}
\label{sec:intro}

A motivation is ...

%\lipsum[2]
	
%% \subsection{Preliminaries}
%% \label{sec:pre}
	
%% \lipsum[3]
	
%% \subsection{Previous Results}
%% \label{sec:prev-results}
	
%% Null graphs are discussed in \cite{HararyR74}
%% The concept of ``internally stable set'' was used in \cite{Berge57, Berge58}.
	
%% \begin{theorem}
%% 	\label{thrm:1}
%% 	\lipsum[4]
%% \end{theorem}
%% \begin{proof}
%% 	content...
%% \end{proof}

%% \begin{corollary}
%% \label{cor:1}
	
%% \lipsum[5]
%% \end{corollary}

%% Unordered List (taken from Overleaf)
%% \begin{itemize}
%% 	\item The individual entries are indicated with a black dot, a so-called bullet.
%% 	\item The text in the entries may be of any length.
%% \end{itemize}

%% Ordered List (taken from Overleaf)
%% \begin{enumerate}
%% 	\item The labels consists of sequential numbers.
%% 	\item The numbers starts at 1 with every call to the enumerate environment.
%% \end{enumerate}

%% \begin{table}[ht]
%% 	\centering
%% 	\begin{tabular}{|c|c|}
%% 		\hline
%% 		\textbf{Odd} & \textbf{Even} \\
%% 		\hline\hline
%% 		One & Two \\
%% 		\hline
%% 		Three & Four \\
%% 		\hline
%% 	\end{tabular}
%% 	\caption{This is a table}
%% 	\label{tbl:1}
%% \end{table}

%% Table~\ref*{tbl:1} is an example of a table.
	
%% \section{More Examples}
%% \label{sec:examples}
	
%% 	%% Now we include a figure.
%% 	%% (See Figure~\ref{fig:example}.)
%% 	%% \begin{figure}[ht]
%% 	%% 	\centering
%% 	%% 	\includegraphics[width=0.3\textwidth]{example}
%% 	%% 	\caption{An example of a figure}
%% 	%% 	\label{fig:example}
%% 	%% \end{figure}

\section{A RPC Calculus for Links}
\label{sec:linksrpc}

\begin{figure}[t]
\centering  
\begin{tabular}{ l  l  r  c  l }
\multicolumn{5}{l}{\textbf{Syntax}} \\
 & Location & $a,b$   & $::=$ & $\client \ \ | \  \  \server$ \\
 &          & $?a$    & $::=$  & $a  \ \ |  \ \ (unknown) $ \\
 & Term     & $L,M,N$ & $::=$  & $V  \ | \  L \ M  \ | \  M[A]  \ | \  (L,M)  \ |  \ \pi_i(M)$ \\
 & Value & $V,W$ & $::=$ & $x  \ \ |  \ \ \lambda^{?a} x.M  \ \ |  \ \ \Lambda\alpha.V  \ \ |  \ \ (V,W)$
\\[\ruleverticalsep]
\multicolumn{5}{l}{\textbf{Types}} \\
& Type & $A,B,C$ & $::=$
& $base  \ \ | \ \  A \rightarrow B  \ \ | \ \  \alpha  \ \ | \ \  A \times B  \ \ | \ \  \forall\alpha.A$ \\[\ruleverticalsep]
\end{tabular}
\caption{The Links-RPC calculus \linksrpc}
\label{fig:linksrpc}
\end{figure}

This section describes an untyped RPC calculus that extends with
unannotated $\lambda$ abstractions whose location is not specified.
%
The untyped and extended RPC calculus intends to capture the RPC
feature of the Links programming language
\cite{Cooper:2006:LWP:1777707.1777724} in a simple way.
%
Let us call it \linksrpc.


The terms and types of the Links-RPC calculus are shown in Figure
\ref{fig:linksrpc}.
%
They are almost the same as those for the polymorphic RPC calculus
\cite{CHOI:scp2020} available in the appendix for reference.
%
In the calculus, it is not mandatory to annotate locations to
$\lambda$ abstractions.
%
$?a$ means that every annotation is either a location constant $a$ or
unknown.
%
There is no location variables used in the calculus.
%
Types are {\it untyped} in terms of specifying locations, which means
that function types are $A \rightarrow B$ with no locations.
%


The semantics for \linksrpc is almost the same as for the polymorphic
RPC calculus \cite{CHOI:scp2020} available in the appendix for
reference.
%
A notable difference is an evaluation rule for unannotated $\lambda$
abstractions that evaluate to $\lambda$ abstractions annotated with
locations of evaluation as

\[
    \begin{prooftree}
      \infer[left label=(UnknownAbs)]0{ \evalRPC{\lamL{}{x}{M}}{a}{\lamL{a}{x}{M} }}
    \end{prooftree}
\]
This evaluation rule was proposed by the original RPC calculus
\cite{Cooper:2009:RC:1599410.1599439}.
%


%% \paragraph{Acknowledgements} \lipsum[6]
	
% \newpage
\bibliography{polyrpc2021}
	
\appendix
	
\section{The Polymorphic RPC Calculus}
\label{app:1}

This section reminds the reader of the polymorphic RPC calculus
\cite{CHOI:scp2020}.
%
It is a polymorphically typed call-by-value $\lambda$-calculus with
location annotations on $\lambda$-abstractions specifying where to
run.
%
The calculus offers the notion of polymorphic location to write
polymorphically located functions succinctly, which is convenient for
programmers.

\subsection{The Syntax and the Semantics}
\label{sec:polyrpc:syntax&semantics}

\begin{figure}[t]
\centering  
\begin{tabular}{ l  l  r  c  l }
\multicolumn{5}{l}{\textbf{Syntax}} \\
 & Location & $a,b$   & $::=$ & $\client \ \ | \  \  \server$ \\
 &          & $\Loc$  & $::=$  & $a  \ \ |  \ \ l$ \\
 & Term     & $L,M,N$ & $::=$  & $V  \ | \  L \ M  \ | \  M[A]  \ | \  M[\Loc]  \ | \  (L,M)  \ |  \ \pi_i(M)$ \\
 & Value & $V,W$ & $::=$ & $x  \ \ |  \ \ \lambda^{Loc} x.M  \ \ |  \ \ \Lambda\alpha.V  \ \ |  \ \ \Lambda l.V \ | \ (V,W)$ \\[\ruleverticalsep]
\multicolumn{5}{l}{\textbf{Semantics}} \\
\end{tabular}

\begin{tabular}{p{\rulewidth} }
  {
    \begin{prooftree}
      \infer[left label=(Abs)]0{ \evalRPC{\lamL{b}{x}{M}}{a}{\lamL{b}{x}{M} }}
    \end{prooftree}
    \ \ \ 
    \begin{prooftree}
      \hypo{ \evalRPC{L}{a}{\lamL{b}{x}{N}} }
      \hypo{ \evalRPC{M}{a}{W} }
      \hypo{ \evalRPC{N\subst{W}{x}}{b}{V}  }
      \infer[left label=(App)]3{ \evalRPC{L \ M}{a}{V}  }
    \end{prooftree}
  }
\\[\ruleverticalsep]
  {
    \begin{prooftree}
      \infer[left label=(Tabs)]0{ \evalRPC{\Lambda\alpha.V}{a}{\Lambda\alpha.V }}
    \end{prooftree}
    \ \ \ \ \
    \begin{prooftree}
      \hypo{ \evalRPC{M}{a}{\Lambda\alpha.V} }
      \infer[left label=(Tapp)]1{ \evalRPC{M[B]}{a}{V\subst{B}{\alpha}} }
    \end{prooftree}
  }
\\[\ruleverticalsep]
  {
    \begin{prooftree}
      \infer[left label=(Labs)]0{ \evalRPC{\Lambda l.V}{a}{\Lambda l.V} }
    \end{prooftree}
    \ \ \ \ \
    \begin{prooftree}
      \hypo{ \evalRPC{M}{a}{\Lambda l.{V}} }
      \infer[left label=(Lapp)]1{ \evalRPC{M[b]}{a}{V\subst{b}{l}}  }
    \end{prooftree}
  }
\\[\ruleverticalsep]
  {
    \begin{prooftree}
      \hypo{ \evalRPC{L}{a}{V} }
      \hypo{ \evalRPC{M}{a}{W} }
      \infer[left label=(Pair)]2{ \evalRPC{(L,M)}{a}{(V,W) }}
    \end{prooftree}
    \ \ \ \ \
    \begin{prooftree}
      \hypo{ \evalRPC{M}{a}{(V_1,V_2)}  \ \ \ i\in\{1,2\}}
      \infer[left label=(Proj-i)]1{ \evalRPC{\pi_i(M)}{a}{V_i}  }
    \end{prooftree}
  }
\end{tabular}
\caption{The polymorphic  RPC calculus \polyrpc}
\label{fig:polyrpc}
\end{figure}

Figure \ref{fig:polyrpc} shows the syntax and semantics of the
polymorphic RPC calculus, {\polyrpc} that allows programmers to use
the same syntax of $\lambda$-application for both local and remote
calls, and allows them to compose differently located functions
arbitrarily.
%
An important feature is the notion of location variable $l$ for which
a location constant $a$ can be substituted.
%
A syntactic object $\Loc$ is either a location constant or a location
variable.
%
Assuming the client-server model in the calculus, location constants
are either $\client$ denoting client or $\server$ denoting server.

In the syntax, $M$ denotes terms, and $V$ denotes values.
%
Every $\lambda$-abstraction $\lamL{\Loc}{x}{M}$ has a location
annotation of $\Loc$.
%
By substituting a location $b$ for a location variable annotation,
$(\lamL{l}{x}{M})\subst{b}{l}$ becomes a monomorphic
$\lambda$-abstraction $\lamL{b}{x}{(M\subst{b}{l})}$.
%
This location variable is abstracted by the location abstraction
construct $\Lambda l.V$, and it is instantiated by the location
application construct $M[\Loc]$.
%
Term applications are denoted by $L \ M$.  The rest of the syntax are
straightforward.

The semantics of {\polyrpc} is defined in the style of a big-step
operational semantics whose evaluation judgments, $\evalRPC{M}{a}{V}$,
denote that a term $M$ evaluates to a value $V$ at location $a$.
%
In the semantics, location annotated $\lambda$-abstractions, type
abstractions, and location abstractions are all values.
%
So, (Abs), (Tabs), and (Labs) are straightforwardly defined as an
identity evaluation relation over them.
%
(App) defines local calls when $a=b$ and remote calls when $a\not=b$
in the same syntax of lambda applications.
%
The evaluation of an application $L \ M$ at location $a$ performs
$\beta$-reduction at location $b$, where a $\lambda$-abstraction
$\lamL{b}{x}{N}$ from $L$ has as an annotation, with a value $W$ from
$M$, and it continues to evaluate the $\beta$-reduced term
$N\subst{W}{x}$, which is a substitution of $W$ for $x$ in $N$, at the
same location.
%
The remaining semantics rules are easily understood.

\begin{figure}[t]
\centering            
\begin{tabular}{l l r c l}
\multicolumn{5}{l}{\textbf{Types}} \\
& Type & $A,B,C$ & $::=$
& $base  \ \ | \ \  A\funL{Loc}B  \ \ | \ \  \alpha  \ \ | \ \  A \times B  \ \ | \ \  \forall\alpha.A 	\ \ | \ \  \forall l.A $ \\
& Type environment & $\Gamma$ & $::=$
& $\emptyset \ \ | \ \ \Gamma, x:A \ \ | \ \ \Gamma, \alpha \ \ | \ \ \Gamma, l$ \\[\ruleverticalsep]
\multicolumn{5}{l}{\textbf{Typing Rules}} \\
\end{tabular}

\begin{tabular}{p{\rulewidth}}
  {
    \begin{prooftree}
      \hypo{  \tyenv(x)=A }
      \infer[left label=(T-Var)]1{ \typing{\tyenv}{\Loc}{x}{A} }
    \end{prooftree}
    \ \ \ \ \
    \begin{prooftree}
      \hypo{ \typing{\tyenvExt{x}{A}}{\Loc}{M}{B} }
      %\hypo{ flv(\Loc) \subseteq \Delta  }
      \infer[left label=(T-Abs)]1{ \typing{\tyenv}{\Loc'}{\lamL{\Loc}{x}{M}}{A\funL{\Loc}B} }
    \end{prooftree}
  }
\\[\ruleverticalsep]
  {
    \begin{prooftree}
      \hypo{  \typing{\tyenv}{\Loc}{L}{A\funL{\Loc'}B } }
      \hypo{  \typing{\tyenv}{\Loc}{M}{A} }
      %\hypo{  flv{A}\cup flv(\Loc') \subseteq \tyenv  }
      \infer[left label=(T-App)]2{ \typing{\tyenv}{\Loc}{L \ M}{B}   }
    \end{prooftree}
  }
\\[\ruleverticalsep]
  {
    \begin{prooftree}
      \hypo{  \typing{\tyenv,\alpha}{\Loc}{V}{A} }
      \infer[left label=(T-Tabs)]1{ \typing{\tyenv}{\Loc}{\Lambda\alpha.V}{\forall\alpha.A}   }
    \end{prooftree}
    \ \ \
    \begin{prooftree}
      \hypo{  \typing{\tyenv}{\Loc}{M}{\forall\alpha.A} }
      \infer[left label=(T-Tapp)]1{ \typing{\tyenv}{\Loc}{M[B]}{A\subst{B}{\alpha}}   }
    \end{prooftree}
  }
\\[\ruleverticalsep]
  {
    \begin{prooftree}
      \hypo{ \typing{\tyenvExtWith{l}}{\Loc}{V}{A} }
      %\hypo{ l \not\in flv(\varenv)\cup flv(\tyenv) \cup flv(\Loc) }
      \infer[left label=(T-Labs)]1{ \typing{\tyenv}{\Loc}{\Lambda l.V}{\forall l.A }}
    \end{prooftree}
    \ \ \
    \begin{prooftree}
      \hypo{ \typing{\tyenv}{\Loc}{M}{\forall l.A } }
      % \hypo{ flv(\Loc')\subseteq\tyenv  }
      \infer[left label=(T-Lapp)]1{ \typing{\tyenv}{\Loc}{M[\Loc']}{A\subst{\Loc'}{l}}}
    \end{prooftree}
  }
\\[\ruleverticalsep]
  {
    \begin{prooftree}
      \hypo{ \typing{\tyenv}{Loc}{L}{A} }
      \hypo{ \typing{\tyenv}{Loc}{M}{B} }
      \infer[left label=(T-Pair)]2{ \typing{\tyenv}{Loc}{(L,M)}{ A \times B }}
    \end{prooftree}
  }
\\[\ruleverticalsep]
  {
    \begin{prooftree}
      \hypo{ \typing{\tyenv}{Loc}{M}{A_1 \times A_2} \ \ \ i\in\{1,2\} }
      \infer[left label=(T-Proj-i)]1{ \typing{\tyenv}{Loc}{\pi_i(M)}{ A_i } }
    \end{prooftree}
  }
\end{tabular}
\caption{A type system for the polymorphic  RPC calculus}
\label{fig:polyrpctysystem}
\end{figure}


\subsection{The Type System}
\label{sec:polyrpc:typesystem}

Figure \ref{fig:polyrpctysystem} shows a type system for the
polymorphic RPC calculus \cite{CHOI:scp2020} that can identify remote
procedure calls at the type level, supporting location polymorphism.
%
The type language allows function types $A \funL{\Loc} B$.
%
Then every $\lambda$-abstraction at unknown location gets assigned
$A\funL{l} B$ using some location variable $l$.
%
A universal quantifier over a location variable, $\forall l. A$, is
also introduced to allow to abstract such occurrences of location
variables.

Typing judgments are in the form of $\typing{\tyenv}{\Loc}{M}{A}$,
saying a term $M$ at location $a$ has type $A$ under a type
environment $\tyenv$.
%
The location annotation, $\Loc$, is either a location variable or
constant.
%
Typing environments $\tyenv$ have location variables, type variables,
and types of variables, as $\{l_1,
\cdots,l_n,\alpha_1,\cdots,\alpha_k, x_1:A_1, \cdots, x_m:A_m\}$.
%
They are used to keep track of a set of free location, type, and value
variables in the context of a given term.


The typing rules for the polymorphic RPC calculus are defined as
follows.
%
(T-App) is a refinement of the conventional typing rule for
$\lambda$-applications with respect to the combinations of location
$\Loc$ (where to evaluate the application) and location $\Loc'$ (where
to evaluate the function).
%
When $\Loc=\Loc'$, one can statically decide that it is a local
procedure call.
%
Otherwise, $\Loc$ is different from $\Loc'$.
%
When both locations are constants as $\Loc=a$ and $\Loc'=b$, $L \ M$
is statically found to be a remote procedure call: if $a=\client$ and
$b=\server$, it is to invoke a server function from the client, and if
$a=\server$ and $b=\client$, it is to invoke a client function from
the server.
%
When at least one of them is a location variable, we cannot make a
decision statically.
%
(T-Labs) and (T-Lapp) are similar to the typing rules for type
abstraction and type application.
%
(T-Labs) checks if its bound location variable does not appear in the
type environment and in the contextual location.
%
(T-Lapp) substitutes $\Loc'$ for all occurrences of a location
variable $l$ on $\lambda$-abstractions in $M$.

The type soundness of the type system for the polymorphic RPC
calculus, which was formulated as Theorem \ref{thm:typesoundness} and
was proved by \cite{CHOI:scp2020}, guarantees that every remote
procedure call thus identified statically will never change to a local
procedure call under evaluation.
%
This enables compilers to generate call instructions for local calls
and network communication for remote calls safely even thogh both are
in the same syntax of lambda applications.

\begin{theorem}[Type soundness for \polyrpc \cite{CHOI:scp2020}]
For a closed term $M$, if \ $\typing{\emptyset}{a}{M}{A}$ and
$\evalRPC{M}{a}{V}$, then $\typing{\emptyset}{a}{V}{A}$.
\label{thm:typesoundness}
\end{theorem}

	
%% \lipsum[7]
	
\end{document}
